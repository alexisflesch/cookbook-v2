\documentclass{book}
\usepackage[T1]{fontenc}
\usepackage[utf8]{inputenc}
\usepackage[french]{babel}
\usepackage{nicefrac}
\usepackage{graphicx}
\usepackage[usenames,dvipsnames]{color}
\usepackage{livrerecettes}
\usepackage{calories}

\usepackage{eso-pic}
\newcommand\BackgroundPic[1]{
\put(0,0){
\parbox[b][\paperheight]{\paperwidth}{%
\vfill
\centering
\includegraphics[width=\paperwidth,height=\paperheight]{#1}%
\vfill
}}}

\usepackage{multind}
\makeindex{cat} %Index par catégories (gâteaux, tartes, viandes, poissons...)
\makeindex{nom} %Index par nom de recettes


%%%%%%%%%%%%%%%% Ajouter une recette + compteur de recettes
\newcounter{nbrecettes}
\newcommand{\ajoute}[3]{%
    \include{#1}%
    \graph[sgraph=photos/petites/#2, bgraph=photos/grandes/#3]%
    \refstepcounter{nbrecettes}%
}

\pagestyle{empty}

%######################## Définition du titre ######################
\definecolor{monbleu}{HTML}{464b6a}
\title{%
\begin{tikzpicture}[overlay]
    \draw[color=white] (0,0) node {~};
    %Coin haut
    \draw[color=orange, line width=4pt] (-8,1) -- (-8,5);
    \draw[color=orange, line width=4pt] (-8.5,4.5) -- (-4,4.5);
    %Mon cahier de recettes
    \draw[color=orange] (-8,2) node[right] {{\bsi{70pt}{75pt}Mon cahier}};
    \draw[color=orange] (8,-1.5) node[left] {{\bsi{70pt}{75pt}de recettes}};
    %Coin bas
    \draw[color=orange, line width=4pt] (4.5,-3.7) -- (9,-3.7);
    \draw[color=orange, line width=4pt] (8.5,-4.2) -- (8.5,-.2);
    %Nombre de recettes
    \draw[color=monbleu] (8,-13.5) node[left] {{\bsi{20pt}{22pt} \ref{nbrecettes} recettes
        pour les gourmands}};
\end{tikzpicture}
}
\author{~}
\date{~}



%%%%%%%%%%%%%%%%%%% TABLE DES MATIÈRES %%%%%%%%%%%%%%%%%%%%%%%%%%%%%%%
%% Control the fonts and formatting used in the table of contents.
\usepackage[titles]{tocloft}
%% Aesthetic spacing redefines that look nicer to me than the defaults.
\setlength{\cftbeforechapskip}{2ex}
\setlength{\cftbeforesecskip}{.5ex}
% Police pour les parties dans la table des matières
\renewcommand{\cftpartfont}{%
  \bsi{18pt}{19pt}
}


%%% INDEXES : pour ne pas qu'ils apparaissent dans la table des matières
\makeatletter
\def\printindex#1#2{\@restonecoltrue\if@twocolumn\@restonecolfalse\fi
  \columnseprule \z@ \columnsep 35pt
  \newpage \twocolumn[{\Large\bf #2 \vskip4ex}]
  \markright{\uppercase{#2}}
  \@input{#1.ind}}
\makeatother


%Jolie table des matières
\makeatletter
\renewcommand*\l@subsubsection{\hfill\newline}
\makeatother


\begin{document}



\frontmatter
\AddToShipoutPicture*{\BackgroundPic{couverture/background_couv.png}}
\maketitle
\renewcommand{\thepage}{}

\renewcommand{\contentsname}{%
\noindent\hspace{-1cm}
\textcolor{monbleu}{\bsi{45pt}{58pt}Table des matières}%
}
\tableofcontents

\mainmatter

% \AddToShipoutPicture{\BackgroundPic{patisserie/background_patisserie.png}}

%%%%%%%%%%%%%%%%%%%%%%%%%%%%%%%%%%% ENTRÉES %%%%%%%%%%%%%%%%%%%%%%%%%%%%%%
\cleardoublepage
\begin{tikzpicture}[overlay]
    \draw[color=white] (0,0) node[right] {~};
    \draw[color=monbleu] (0.5,-7) node[right] {{\bsi{50pt}{58pt}Première partie}};
    \draw[color=monbleu] (3.2,-9.5) node[right] {{\bsi{50pt}{58pt}Entrées}};
\end{tikzpicture}
\renewcommand{\thepage}{}
\addcontentsline{toc}{part}{Entrées \bigskip}


%%%%%%%%%%%%%%%%%%%%%%%%%%%%%%%%%% PLATS CHAUDS %%%%%%%%%%%%%%%%%%%%%%%%%%%%%%
\cleardoublepage
\begin{tikzpicture}[overlay]
    \draw[color=white] (0,0) node[right] {~};
    \draw[color=monbleu] (0.5,-7) node[right] {{\bsi{50pt}{58pt}Deuxième partie}};
    \draw[color=monbleu] (1.5,-9.5) node[right] {{\bsi{50pt}{58pt}Plats chauds}};
\end{tikzpicture}
\renewcommand{\thepage}{}
\addcontentsline{toc}{part}{Plats Chauds \bigskip}%


%%%%%%%%%%%%%%%%%%%%%%%%%%%%%%%%% PATISSERIE %%%%%%%%%%%%%%%%%%%%%%%%%%%%%%
\cleardoublepage
\begin{tikzpicture}[overlay]
    \draw[color=white] (0,0) node[right] {~};
    \draw[color=monbleu] (0.5,-7) node[right] {{\bsi{50pt}{58pt}Troisième partie}};
    \draw[color=monbleu] (2.5,-9.5) node[right] {{\bsi{50pt}{58pt}Pâtisserie}};
\end{tikzpicture}
\renewcommand{\thepage}{}
\addcontentsline{toc}{part}{Pâtisserie \bigskip}%


\ajoute{recettes/patisserie/ShortBread}{shortbread}{beurre}

%%%%%%%% Nombre de recettes
\label{nbrecettes}

%%%%%%%%%%%%%%%%%%%%%%%%%% Annexes %%%%%%%%%%%%%%%%%%%%%%%%%%%%%%%
\backmatter
\cleardoublepage

\begin{tikzpicture}
    \draw[color=white] (0,0) node[right] {~};
    \draw[color=myblue] (0.5,-7) node[right] {{\bsi{50pt}{58pt}Annexes}};
\end{tikzpicture}
\renewcommand{\thepage}{}


\printindex{nom}{\bsi{20pt}{25pt}Index alphabétique des recettes}
\printindex{cat}{\bsi{20pt}{25pt}Index par catégories des recettes}


\end{document}
